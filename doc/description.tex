\documentclass[12pt,a4paper]{article}
\usepackage[utf8]{inputenc}
\usepackage[russian]{babel}
\usepackage{pscyr}
\usepackage{amsmath}
\usepackage{amsfonts}
\usepackage{amssymb}
\usepackage{makeidx}
\usepackage[left=3cm, right=3cm, top=1cm, bottom=2cm]{geometry}
\usepackage{graphicx}

\begin{document}

\newcommand{\parw}[2]{\dfrac{\partial #1}{\partial #2}}
\newcommand{\pl}{p_{\lambda}}
\newcommand{\pa}{p_{a}}
\newcommand{\pgm}{p_{m, \sigma^2}}
\newcommand{\pma}{p_{M, a}}
\newcommand{\xseqn}{x_1, \ldots, x_n}
\renewcommand{\le}{\leqslant}


\binoppenalty=10000
\relpenalty=10000
\newcommand*{\hm}[1]{#1\nobreak\discretionary{}%
{\hbox{$\mathsurround=0pt #1$}}{}}

\begin{center}
\textbf{Домашняя работа 1. Отчет}
\end{center}

\begin{flushright}
\textit{Богатый Антон}

\textit{Лапин Александр}
\end{flushright}


\section{Постановка задачи:}
\indent\indent
Дан список итальянских слов, для каждого слова необходимо определить формой каких слов оно является, указать их формы и части речи. При этом, мы рассматриваем только глаголы, существительные и прилагательные. Качество результатов оценивалось по усредненной по всем словам F1 норме.
\bigskip

\section{Метод решения:}
\indent\indent
Было предложено собрать морфологический трансдьюсер, используя один из FST-тулкитов: \texttt{foma} или \texttt{SFST}.
Мы решили, что туториал по \texttt{foma} недостаточно полно описывает производимые в нем действия и содержит ряд ошибок, поэтому выбор пал на \texttt{SFST}. За основу программы был взят пример английской морфологии из соответствующего туториала.


\subsection{Глаголы}
Первой проблемой стала необходимость разобраться с идеей работы данного тулкита, а далее --- понять что с ним делать, т.к. задание неколько отличается от задачи, разбираемой в примере. Далее, пользуясь советом, мы начали с глаголов. После реализации правил образования различных форм глаголов в различных временах и запуске его на глаголах из тестовой выборки, трансдьюсер был оценен на $0.30$ баллов по F1 норме. Последующими изменениями этот показатель был доведен до $0.65$. Что было сделано?

\begin{enumerate}
\item Изначально наш трансдьюсер не учитывал форм глаголов, требующих вспомогальных слов. Были добавлены окончания для соотвествующих времен.
\item Мы отсортировали нераспознанные слова по окончаниям и поняли, что окончаний, предлагаемых на указаном в задании сайте, явно не хватает. Поэтому были приняты попытки найти другие формы глаголов на сторонних ресурсах.
\item Неторые правила образования форм так и не удалось найти, зато их было нетрудно определить по данной выборке. К сожалению, мы не знаем как называются эти формы (например, когда глагол пишется без окончания или без последней гласной (-ar, -er, -ir)), но правила были добавлены :)
\item Было замечено правило, по которому, если глагол оканчивается на $-i[aei]re$, то в некоторых формах $i$ пропадает. Было добавлено. Правда, точного правила мы не знаем.
\item Аналогично, были добавлены правила замены в окончании сдвоенных $nn$ на $n$, отбрасывания первой гласной окончания, вставки $h$ перед окончанием (хотя, вроде $h$, вставляется в основном после $c$, а, возможно, даже сдвоенной $c$).
\end{enumerate}

Все эти добавления позволили улучшить результат, хотя таким образом трансдьюсер начал обрабатывать множество некорректно написанных форм глаголов :( Что не было учтено:

\begin{enumerate}
\item Оставшиеся "непонятные" окончания
\item Неправильные глаголы
\end{enumerate}

\subsection{Существительные}
\indent\indent
После работы с глаголами структура существительных показалась более простой. После реализации правил образования различных форм было выявлено, что в выборке встречается довольно много иностранных (относительно итальянского) слов (английские, французские, немецкие?). Поэтому трансдьюсер для всех слов пробует образовать форму множественного числа по правилу для английского языка (добавлением $s$ и $es$), но это, наверное, нехорошо. Оценка на существительных также оказалась порядка $0.65$.

Что не было учтено:
\begin{enumerate}
\item Формы составных существительных (когда окончание изменяется у первого слова)
\item Существительные-исключения
\item Заимствованные слова
\end{enumerate}

\subsection{Прилагательные}
\indent\indent
Как и в случае с глаголами, правил на предлагаемом сайте оказалось недостаточно. Некоторые прилагательные очень похожи на глаголы (оканчиваются на $ire$, $ere$, $are$), что удалось заметить после сортировки по окончаниям неопознанных слов. Результат был тоже на уровне $0.65$. Неучтенными пунктами остались пункты, аналогичные уже сказанным.

\subsection{Словарь}
\indent\indent
После запуска реализованного трасдьюсера на тестовой выборке, оказалось, что прилагаемого словаря явно недостаточно. Трансдьюсер определил лишь жалкую долю слов :( 

Мы взяли словарь из \texttt{OpenOffice}. В нем слова оканчивались какими-то символами, обозначающими принадлежность слова к какой-либо части речи и т.п. (из серии /ATXpI), а инструкции по этим окончаниям не было. Мы выписали из этого словаря слова из обучающей выборки (отдельно глаголы, отдельно существительные и отдельно прилагательные), потом оставили от данных слов только окончания и выписали слова с соотвествующими окончаниями в соответствующие словари глаголов, существительных и прилагательных. Да, это, наверное, не самый лучший способ, к тому же в итоге эти словари, вроде, довольно сильно пересекаются).

\subsection{Кодировка}
\indent\indent
По ходу работы возникали различные проблемы с символами с черточками. У глаголов, казалось, черточка всегда направлена в одну сторону, а у существительных оказались наверное французские слова с черточкой в другую сторону. Пришлось добавлять в список букв. Также, в тестовой выборке есть слово с $i$ с двумя точками, её мы добавлять не стали. Изначально все текстовые файлы были переведены в кодировку \texttt{UTF8}, а в итоге, при попытке сдать полученный ответ в тестирующую систему, возникли некоторые трудности --- пришлось ручками переводить всё в кодировку \texttt{cp1252}, исправляя неправильно преобразованные символы с черточками.

\section{Результаты:}
\indent\indent
Тестирующая система оценила наш трансдьюсер на $0.79$ баллов, чему мы очень обрадовались, т.к. рассчитывали на $0.30$ или $0.60$. Можно было бы подправить некоторые вещи, но очень не хочется заново преобразовывать кодировки ручками, возможно мы что-то делаем не так :( Нами не были учтены слова-исключения, некоторые формы слов, сложные слова и некоторые правила, но, при необходимости, возможен вывод лица/числа/рода/времени формы слова, что, на наш взгляд, является полезным свойством нашей реализации. Мы уверены, что после корректировки словаря и правил, можно получить более хороший результат.

\end{document}


